% 
% 1章 はじめに
%
\chapter{はじめに}

\util は\cmml を\tac で実行するために必要な5つのツールから構成されます。


\begin{itemize}
\item {\as}は\tac 用のアセンブラです。
アセンブラは{\cmm}コンパイラが出力したアセンブリ言語プログラムを、
リロケータブルオブオブジェクトに変換します。

アセンブリ言語の文法は、
「付録\ref{app:as} {\as}文法のまとめ」に簡単にまとめてあります。
現在のところ詳しいドキュメントがありません。
「文法のまとめ」の他には、{\as}の動作テストに使用される
\verb;Util--/AS--/test.s;ファイルが参考になります。

リロケータブルオブジェクトは、
機械語プログラム、名前表、再配置情報表からなるファイルです。
詳しくは、「付録\ref{app:oformat} {\tt .o} 形式ファイル」で説明します。

\item {\ld}は\tac 用のリンカです。
リンカは複数のリロケータブルオブジェクトを入力して、
一つのリロケータブルオブジェクトに結合します。

\item {\objexe}は、
リロケータブルオブジェクトを入力し、
{\tacos}のアプリケーションプログラムの実行形式ファイルを出力します。
実行形式ファイルは、リロケータブルオブジェクトから、
シンボルテーブルなどリンカしか使用しない情報を取り除いたファイルです。
実行形式ファイルについては「付録\ref{app:eformat}
{\tt .exe} 形式ファイル」で説明します。

\item {\objbin}は\tac 用のローダです。
ローダはリロケータブルオブジェクトを入力して、
ロードアドレスが決定された機械語を出力します。
出力ファイルの形式については「付録\ref{app:bformat}
{\tt .bin} 形式ファイル」で説明します。
ローダは、{\tacos}のカーネルを作成する時に使用されます。

\item {\size}は、
リロケータブルオブジェクトのテキストセグメント、
初期化データセグメント、
非初期化データセグメントの大きさを表示します。
完成したプログラムのメモリ使用量を見積もるために使用します。
\end{itemize}
