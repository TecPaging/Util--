% 
% $Id$
%
\chapter{コマンドリファレンス}

\util に含まれる5つのツールの使用方法を解説します。

\section{{\as}コマンド}

{\as}コマンドは{\tac}用のアセンブラです。
{\tac}のアセンブリ言語で記述されたプログラムを
リロケータブルオブジェクトに変換します。
リロケータブルオブジェクトファイルの形式は、
「付録\ref{app:oformat} \verb/.o/形式ファイル」で解説してあります。

{\as}には手書きのプログラムを入力することもできますが、主に、
{\cmm}コンパイラが出力したアセンブリ言語プログラムを
入力することを想定しています。
そのため、オペランドにアドレス計算式が書けない等、
{\cmm}コンパイラが使用しない機能は省略されています。
{\tac}アセンブリ言語の文法は「付録\ref{app:as} \as 文法まとめ」の通りです。

{\as}コマンドの書式は次の通りです。

\begin{flushleft}
{\bf 形式 : }~~~\verb/as-- [-h] [-v] [<ソースプログラムファイル>]/
\end{flushleft}

\verb/-h/、\verb/-v/オプションは使用法メッセージを表示します。
ソースプログラムファイルの拡張子は「\verb/.s/」でなければなりません。
{\as}コマンドは拡張子が「\verb/.o/」のファイルを作成し、
再配置可能な機械語を出力します。
次のように実行すると、``\verb/hello.o/''ファイルが作成されます。

\begin{mylist}
\begin{verbatim}
$ as-- hello.s
\end{verbatim}
\end{mylist}

\section{{\ld}コマンド}

{\ld}コマンドは{\tac}用のリンカープログラムです。
「\verb/.o/形式ファイル」(\pageref{app:oformat}ページ参照)を複数入力し、
同じ形式の一つのファイルに結合します。
{\ld}コマンドの書式は次の通りです。
出力ファイルは一つだけ、
入力ファイルはいくつでも指定できます。

\begin{flushleft}
{\bf 形式 : }~~~\verb/ld-- [-h] [-v] <出力ファイル> <入力ファイル> .../
\end{flushleft}

次のように実行すると``\verb/libtac.o/''、``\verb/hello.o/''の
二つのファイルを結合して、``\verb/mod.o/''ファイルを作成します。
``\verb/hello.sym/''ファイルは出力されたリロケータブルオブジェクトの
再配置表と名前表をダンプしたファイルです。

``\verb/mod.o/''ファイルもリロケータブルオブジェクトなので、
未定義シンボルを含んでいる可能性があります。

\begin{mylist}
\begin{verbatim}
$ ld-- mod.o /usr/local/cmmLib/libtac.o hello.o > hello.sym
\end{verbatim}
\end{mylist}

\section{{\objexe}コマンド}

{\objexe}コマンドは{\tacos}用の実行形式ファイル作成プログラムです。
「\verb/.o/形式ファイル」(\pageref{app:oformat}ページ参照)を入力し、
「\verb/.exe/形式ファイル」(\pageref{app:eformat}ページ参照)へ変換します。
{\objexe}コマンドの書式は次の通りです。
入力ファイルも出力ファイルも一つだけ指定できます。

\begin{flushleft}
{\bf 形式 : }~~~\verb/objexe-- [-h] [-v] <exefile> <objfile> <stkSiz>/
\end{flushleft}

``\verb/<exefile>/''は出力のファイル名です。
``\verb/<objfile>/''は入力のファイル名です。

``\verb/<stkSiz>/''には、
アプリケーションプログラムに割り付ける
「スタック領域サイズ+ヒープ領域サイズ」をバイト単位の10進数で指定します。

下に実行例を示します。
\verb/mod.o/は、\ld が出力したリロケータブルオブジェクトです。
\verb/hello.map/ファイルには、
\verb/hello.exe/中のシンボルとアドレスの対応表がアドレス順に格納されます。
シンボルのアドレスは、アプリケーションプログラムの先頭からの相対アドレスです。

\begin{mylist}
\begin{verbatim}
$ objexe-- hello.exe mod.o 600 | sort --key=1 > hello.map
\end{verbatim}
\end{mylist}

\verb/mod.o/に未解決シンボルが含まれているとエラーになります。
次の実行例は\verb/hello.s/ファイル
(\verb/hello.cmm/から\cmmc が変換したアセンブリ言語ソースプログラム)中に、
未定義シンボル\verb/_printff/(\cmm ソース中では\verb/printff/)が
含まれていてエラーになった例です。

\begin{mylist}
\begin{verbatim}
$ objexe-- hello.exe mod.o 600 | sort --key=1 > hello.map
_printff[hello.s]:undefined symbol
\end{verbatim}
\end{mylist}

\section{{\objbin}コマンド}

{\objbin}コマンドは{\tac}用のローダプログラムです。
「\verb/.o/形式ファイル」(\pageref{app:oformat}ページ参照)を入力し、
「\verb/.bin/形式ファイル」(\pageref{app:bformat}ページ参照)へ変換します。
{\objbin}コマンドの書式は次の通りです。
入力ファイルも出力ファイルも一つだけ指定できます。

\begin{flushleft}
{\bf 形式 : }~~~\verb/objbin-- [-h] [-v] 0xTTTT <出力ファイル> <入力ファイル> [0xBBBB]/
\end{flushleft}

\verb/0xTTTT/は16進数でテキストセグメントの先頭アドレスを指定します。
指定した番地からテキストセグメント、データセグメントの順に配置されます。
\verb/0xBBBB/は16進数でBSSセグメントの先頭アドレスを指定します。
\verb/0xBBBB/は省略可能です。
省略した場合、BSSセグメントはデータセグメントの直後に配置されます。

下に実行例を示します。
実行例の1行目は``\verb/kernel.o/''ファイルを入力し\verb/0000H/番地に配置した
状態の機械語を作成し``\verb/kernel.bin/''ファイルに出力します。
実行例の2行目は``\verb/ipl.o/''ファイルを入力し\verb/F000H/番地に
テキストセグメントとデータセグメントを連続して配置し、
\verb/DA00H/番地に BSSセグメントを配置します。
そして、その状態の機械語を``\verb/ipl.bin/''ファイルに出力します。
なお、これらの例は実物のカーネルとIPLを配置する手順です。
\verb/F000H/番地からの始まる ROM 領域に IPL プログラム、
RAM 領域の\verb/DA00H/番地にワークエリアを配置します。

\begin{mylist}
\begin{verbatim}
$ objbin-- 0x0000 kernel.bin kernel.o | sort --key=1 > kernel.map
$ objbin-- 0xf000 ipl.bin ipl.o 0xda00 | sort --key=1 > ipl.map
\end{verbatim}
\end{mylist}

{\objbin}コマンドは、固定番地に配置された完全な機械語を作成します。
``\verb/kernel.o/''ファイルや``\verb/ipl.o/''ファイルに未定義シンボルが
含まれていると完全な機械語に変換できないのでエラーが発生します。

%次の実行例はエラーが発生する例です。
%実行例のエラーメッセージは、
%``\verb/kernel.s/''ファイル中で参照された \verb/_printf/
%と言うシンボルが未定義であることを表しています。
%
%\begin{mylist}
%\begin{verbatim}
%$ objbin-- 0x0000 kernel kernel.o
%_printf[kernel.s]:undefined symbol
%\end{verbatim}
%\end{mylist}

\section{{\size}コマンド}

{\size}コマンドは
「\verb/.o/形式ファイル」(\pageref{app:oformat}ページ参照)または
「\verb/.exe/形式ファイル」(\pageref{app:eformat}ページ参照)を入力し、
テキスト、データ、BSSセグメントのサイズを表示します。

\begin{flushleft}
{\bf 形式 : }~~~\verb/size-- [-h] [-v] <入力ファイル>/
\end{flushleft}

次のように実行します。
``\verb/kernel.o/''ファイルの各セグメントと全体のサイズが
10進数と16進数で表示されています。

\begin{mylist}
\begin{verbatim}
$ size-- kernel.o
text    data    bss     dec     filename
17436     516    4386   22338   kernel.o
(441c)  (0204)  (1122)  (5742)  (hex)
\end{verbatim}
\end{mylist}
