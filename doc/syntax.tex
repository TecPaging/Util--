% 
% $Id$
%
\chapter{{\as}文法まとめ}
\label{app:as}

次のページに、{\as}の文法を BNF 風にまとめたものを掲載します。

\begin{verbatim}
注意1: 次ページで用いる文法表記方法
 全角記号がメタ文字、意味は次の通り
 (1) A ☆   はAのゼロ回以上の繰り返し
 (2) 《...》はグループ
 (3) 【...】は省略可能
 (4) A | B は A または B


注意2: 次ページを読む上での注意事項
 (※0)  ラベルを宣言する時は空白なしに行頭から書く.
 (※1)  ラベル行はテキストセグメントのラベルを定義する.
 (※2)  EQUラベルは定数値(オブジェクトに出力されない).
 (※3)  STRINGデータはTEXTセグメントの最後尾に出力される.
 (※4)  ラベルに予約語と同じ綴りは使用できない.
         '.'で始まるラベルはファイル内のローカルラベルになる.
\end{verbatim}

{\small\tt
\begin{tabular}{l l l}
プログラム:  & 《空行|ラベル行|擬似命令行|機械語命令行》☆ &       \\
空行:        & 【コメント】 \n                                &       \\
ラベル行:    & ラベル 【コメント】 \n                         & (※1) \\
コメント:    & ; 文字☆                                       &       \\
擬似命令行:  & EQU命令行|STRING命令行|DB命令行|            &       \\
              & DW命令行|BS命令行|WS命令行                   &       \\
機械語命令行:& 【ラベル】 機械語記述 \n                       &       \\
EQU命令行     & ラベル EQU 数値 \n                             & (※2) \\
STRING命令行  & 【ラベル】 STRING 文字列 \n                    & (※3) \\
DB命令行      & 【ラベル】 DB 式 \n                            &       \\
DW命令行      & 【ラベル】 DW 式 \n                            &       \\
BS命令行      & ラベル BS 数値 \n                              &       \\
WS命令行      & ラベル WS 数値 \n                              &       \\
機械語記述:  & 機械語1|機械語2|機械語3|機械語4|機械語5|機械語6 & \\
機械語1:     & 命令1                                          &       \\
機械語2:     & 命令2 レジスタ                                 &       \\
機械語3:     & 命令3 レジスタ , 式                      |    &       \\
              & 命令3 レジスタ , @ レジスタ              |    &       \\
              & 命令3 レジスタ ,【 \verb/%/ 】 レジスタ        &       \\
機械語4:     & 命令4 レジスタ , 式                      |    &       \\
              & 命令4 レジスタ , 式 , レジスタ           |    &       \\
              & 命令4 レジスタ , \verb/#/ 式             |    &       \\
              & 命令4 レジスタ , レジスタ                |    &       \\
              & 命令4 レジスタ , @ レジスタ              |    &       \\
              & 命令4 レジスタ , \verb/%/ レジスタ             &       \\
機械語5:     & 命令5 レジスタ , 式                      |    &       \\
              & 命令5 レジスタ , 式 , レジスタ           |    &       \\
              & 命令5 レジスタ , @ レジスタ              |    &       \\
              & 命令5 レジスタ , \verb/%/ レジスタ             &       \\
機械語6:     & 命令6 式                                 |    &       \\
              & 命令6 式 , インデクスレジスタ            |    &       \\
              & 命令6 \verb/%/ レジスタ                        &       \\
レジスタ:    & G0 | G1 | ... | G12 | FP | SP | USP | PC&       \\
命令1:       & NO | RET | RETI | EI | DI | SVC | HALT     &       \\
命令2:       & PUSH | POP                                    &       \\
命令3:       & IN | OUT                                      &       \\
命令4:       & LD|ADD|SUB|CMP|AND|OR|XOR|ADDS|MUL|   &       \\
              & DIV|MOD|MULL|DIVL|SHLA|SHLL|SHRA|SHRL   &       \\
命令5:       & ST                                             &       \\
命令6:       & JZ| JC| JM| JO| JNZ| JNC| JNM| JNO|    &       \\
              & JNO| JLT| JLE| JGE| JGT| JMP| CALL       &       \\
式:          & 数値 | ラベル                                 &       \\
ラベル:      & 英字 英数字☆                                  & (※4) \\
数値:        & 10進数値 | 16進数値 | 8進数値 | 文字コード  &       \\
10進数値:    & 【-】 1..9  0..9☆                             &       \\
16進数値:    & 【-】 0x 16進数字 16進数字☆                   &  \\
8進数値:     & 【-】 0 0..7☆                                 &       \\
文字コード:  & ' 文字 '                                       &       \\
文字列:      & " 文字☆ "                                     &       \\
英字:        & A..Z | a..z | \_ | .                        &       \\
英数字:      & 英字 | 0..9                                   &       \\
16進数字:    & 0..9  | A..F | a..f                          &       \\
\end{tabular}
}
